\section{Komplexe Zahlen}
\sectionauthor{Natalie Teplitska, Ole Fleck, (Mara Germann)}
Da in der Radioastronomie die Messdaten komplexe Zahlen sind, hat sich unser Kurs auch mit diesen beschäftigt.

Die Menge $\mathbb{C}$ ist die Menge aller Zahlen $z=a+b\cdot i$ für $a, b \in \mathbb{R}$. Dabei wird $a$ als Realteil und $b$ als Imaginärteil bezeichnet. Weiterhin ist $i$ die imagin\"are Einheit und definiert über die Eigenschaft: $i^{2} = -1$. Für $b = 0$ erhalten wir die bereits bekannten reellen Zahlen. Man kann $\mathbb{C}$ somit als zweidimensionalen reellen Vektorraum interpretieren.

Die komplex konjugierte von $z=a+b\cdot i$ ist definiert als $\bar{z}=a-b\cdot i$. Das Produkt aus $z$ und $\bar{z}$ hat die Eigenschaft, dass $z\cdot \bar{z}=(a+b\cdot i)\cdot (a-b\cdot i)=a^2-b\cdot i^2=a^2+b^2$ eine reelle Zahl ist.

Zwei komplexe Zahlen werden addiert, indem ihre Real- und Imaginärteile separat addiert bzw. subtrahiert werden. Auch die Multiplikation funktioniert ähnlich wie in $\mathbb{R}$, es sollte jedoch stets bedacht werden, dass $i\cdot i=-1$.Die Division von komplexen Zahlen gestaltet sich etwas komplizierter. Hier hilft es oft, mit der konjugierten komplexen Zahl des Nenners zu erweitern, sodass im Nenner kein Imaginärteil mehr vorkommt und der Ausdruck sich so erheblich vereinfacht.

Die Menge $\mathbb{C}$ lässt sich geometrisch in ein einem kartesischen Koordinatensystem darstellen, wobei die x-Achse für a und die y-Achse für b steht. Jede komplexe Zahl $z=a+b\cdot i$ entspricht dann genau einem Vektor $\begin{pmatrix} a\\b
\end{pmatrix}$.
Bisher haben wir mit den komplexen Zahlen in kartesischen Koordinaten gerechnet. Sie lassen sich aber auch als Polarform angeben:
\begin{displaymath}
z=r\cdot e^{i\cdot\phi}.
\end{displaymath}
Dabei ist $\phi$ der Winkel des Vektors zur x-Achse und r seine Länge. Bei der Multiplikation solcher Vektoren werden Winkel addiert und Längen multipliziert. Eine konjugierte komplexe Zahl ist dadurch äquivalent zur Spiegelung des Vektors an der reellen Achse.