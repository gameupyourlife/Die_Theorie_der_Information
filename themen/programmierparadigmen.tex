
\section{Programmierparadigmen}
\sectionauthor{Alexander Gitnik, Clemens Ljungh, (Jonas Fiedler, Aaron Gschwendt)}
\subsection{Objektorientierte Programmierung}
Die objektorientierte Programmierung ist eins der am weitesten verbreiteten Programmierparadigmen. Beim objektorientierten Programmieren ist das ausschlaggebende Kriterium, dass man den Code zu sinnvollen Strukturen zusammenfasst. Dafür erstellt man Klassen, welche oftmals reale Objekte repräsentieren. 

Eine Klasse ist eine allgemeine Vorlage, aus der einzelne Objekte, die \emph{Klasseninstanzen}, initialisiert werden können. Diese Klassen besitzen zum einen Eigenschaften, \emph{Attribute} genannt, und zum anderen Handlungsmöglichkeiten, welche \emph{Methoden} genannt werden. Jede Klasse besitzt einen Konstruktor, welcher bei der Erstellung einer \emph{Instanz} der Klasse aufgerufen wird. Zudem werden gegebenenfalls die zum Erstellen einer Klasse benötigten Parameter übergeben. Objekte werden durch Zuweisung des Konstruktoraufrufs, der jeweiligen Klasse, an eine Variable initialisiert, welche nun zu einem Objekt des Klassentyps wird. Wenn sich Klassen ähneln, können gleiche \emph{Attribute} und \emph{Methoden} in einer \emph{Superklasse} definiert werden. Diese vererben den Klasseninhalt an die \emph{Unterklassen}, welche die Merkmale übernehmen.

Typischerweise werden \emph{Attribute} lediglich über \emph{Getter-} und \emph{Setter-Methoden} abgefragt und verändert, damit der Zustand des Objekts immer nur kontrolliert verändert werden kann. Dies hat den Grund, dass man nicht versehentlich einen Fehler in den inneren Zustand des Objekts einbauen möchte. Eine \emph{Setter-Methode}, kann überprüfen, welche Werte ein Attribut annimmt. Um eine Methode aufrufen zu können ist es im Normalfall nötig dies über eine bereits initialisierte Instanz der Klasse zu tun. Ein spezieller Dekorator macht die Methode statisch, was bedeutet, dass die Methode über die Klasse aufgerufen werden kann. Dies hat den Vorteil, dass man nicht erst ein Objekt initialisieren muss un die Methode nutzen zu können.

\subsection{Funktionale Programmierung}
Die funktionale Programmierung ist ein gänzlich anderes Konzept als die objektorientierte Programmierung. Funktionale Programmierung hat ihren Ursprung im Lambda-Kalkül. Das Lambda-Kalkül ist eine formale Sprache, um logische Systeme darzustellen. Es ist Turing-vollständig, dies bedeutet, dass man jedes Programm damit ausdrücken kann.

Anders als bei der objektorientierten Programmierung werden Daten bei der funktionalen Programmierung nicht in Objekten gespeichert, sondern ausschließlich durch einzelne Funktionen verarbeitet. Diese Funktionen müssen dabei \emph{pur} sein. \emph{Pure Funktionen} definieren sich dadurch, dass sie bei gleicher Eingabe immer die gleiche Ausgabe geben, da sie keinen inneren Zustand haben, welcher die Eingabe verändern könnte. Weiterhin ändern die puren Funktionen auch nichts außerhalb der eigenen Funktion.

Funktionen können als Parameter den Rückgabewert einer anderen Funktion nehmen, sodass jene ineinander verschachtelt werden. So entsteht in der funktionalen Programmierung natürlicherweise eine Kette aus verschachtelten Befehlen. Häufig wird in der funktionalen Programmierung \emph{Rekursion} verwendet, also Funktionen, die sich selbst aufrufen.

Daten sind bei der objektorientierten Programmierung zwar besser strukturiert, aber dafür sind gerade die puren Funktionen der funktionalen Programmierung einfacher für Entwickler zu verstehen, da die puren Funktionen keinen möglicherweise komplizierten inneren Zustand haben. Wichtig für die funktionale Programmierung ist das Konzept der \emph{Lazy Evaluations}. Dies bedeutet, dass Ausdrücke, erst dann ausgerechnet werden, wenn sie für ein Ergebnis benötigt werden. Dadurch kann in manchen Fällen die Laufzeit eines Programmes verringert werden, was vorteilhaft ist.

Für unsere Kursarbeit verwenden wir oft funktionale Programmierung, da wir Datensätze und probabilistische Modelle rein mathematisch verarbeiten. Außerdem ist Python gut für die funktionale Programmierung geeignet.

\subsection{Gegenüberstellung}
Insgesamt unterscheidet sich damit die objektorientierte Programmierung von der funktionalen Programmierung bereits in ihrem Ansatz. Sie ist realtitätsbezogen, wohingegen die funktionalen Programmierung lediglich an der Programmeffektivität orientiert ist. Die Umsetzung ist daher sehr verschieden, sodass in der objektorientierten Programmierung Klassen und globale Variablen gefordert werden, während die funktionale Programmierung ebendiese verbietet, und mittels puren Funktionen den Code in kleinstmögliche und unabhängige Abschnitte unterteilt.
